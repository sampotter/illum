\documentclass{article}

\usepackage{amsmath}
\usepackage{amsthm}
\usepackage{cleveref}
\usepackage{fullpage}

\begin{document}

\section{Radiosity Methods}

\emph{Make an argument for using the radiosity method}:
\begin{itemize}
\item Radiosity methods fell out of favor \emph{because}...
\item Ideal scenario:
  \begin{itemize}
  \item we don't have to deal with participating media
  \item we actually are doing thermal modeling
  \item the object that we're modeling is ``convex + craters'', so the
    visibility matrix is sparse
  \item we have an object and many point sources at infinity
  \end{itemize}
\end{itemize}

An approach to solving the radiative transfer equation would be to
discretize the integral equation and solve it using one of a variety
of methods. Here are a few ideas and references:
\begin{itemize}
\item Nested dissection
\item Hanrahan~\cite{hanrahan1991rapid}... Inspired by the fast
  multipole method. Probably can't use without some modification
\item Algebraic multigrid
\end{itemize}

Each of these uses a hierarchical approach in some way. We'll need a
way of progressively decimating our meshes while keeping a mapping
which will allow us to rapidly transfer from parent to child
elements. It looks like CGAL has a way to do this (search ``surface
mesh simplication''). There are a bunch of papers which implement this
in one way or another~\cite{khodakovsky2000progressive}. It probably
doesn't matter too much which approach we take. The key is that it's
fast, simple to work with, and doesn't introduce significant error.

\section{General Global Illumination}

\emph{Using BRDFs...}

\bibliographystyle{plain}
\bibliography{illum}

\end{document}

%%% Local Variables:
%%% mode: latex
%%% TeX-master: t
%%% End:
